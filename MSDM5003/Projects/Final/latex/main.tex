\documentclass[runningheads]{llncs}

\usepackage{algorithm}
\usepackage{csvsimple}
\usepackage{algpseudocode}
\usepackage{amsmath}
\usepackage{graphicx}
\usepackage[T1]{fontenc}
\usepackage{beramono}
\usepackage{listings}
\usepackage[usenames,dvipsnames]{xcolor}
\usepackage[colorlinks=true, linkcolor=blue]{hyperref}

%%
%% Julia definition (c) 2014 Jubobs
%%
\lstdefinelanguage{Julia}%
  {morekeywords={abstract,break,case,catch,const,continue,do,else,elseif,%
      end,export,false,for,function,immutable,import,importall,if,in,%
      macro,module,otherwise,quote,return,switch,true,try,type,typealias,%
      using,while},%
   sensitive=true,%
   alsoother={$},%
   morecomment=[l]\#,%
   morecomment=[n]{\#=}{=\#},%
   morestring=[s]{"}{"},%
   morestring=[m]{'}{'},%
}[keywords,comments,strings]%

\lstset{%
    language         = Julia,
    basicstyle       = \ttfamily,
    keywordstyle     = \bfseries\color{blue},
    stringstyle      = \color{magenta},
    commentstyle     = \color{ForestGreen},
    showstringspaces = false,
}

\definecolor{codegreen}{rgb}{0,0.6,0}
\definecolor{codegray}{rgb}{0.5,0.5,0.5}
\definecolor{codepurple}{rgb}{0.58,0,0.82}
\definecolor{backcolour}{rgb}{0.95,0.95,0.92}

\lstdefinestyle{mystyle}{
    backgroundcolor=\color{backcolour},   
    commentstyle=\color{codegreen},
    keywordstyle=\color{magenta},
    numberstyle=\tiny\color{codegray},
    stringstyle=\color{codepurple},
    basicstyle=\ttfamily\footnotesize,
    breakatwhitespace=false,         
    breaklines=true,                 
    captionpos=b,                    
    keepspaces=true,                 
    numbers=left,                    
    numbersep=5pt,                  
    showspaces=false,                
    showstringspaces=false,
    showtabs=false,                  
    tabsize=2
}

\lstset{style=mystyle}

% Used for displaying a sample figure. If possible, figure files should
% be included in EPS format.
%
% If you use the hyperref package, please uncomment the following line
% to display URLs in blue roman font according to Springer's eBook style:
% \renewcommand\UrlFont{\color{blue}\rmfamily}

\begin{document}
%
\title{Report: KPR Game for
Resource Allocation in the Internet of Things}
%
%\titlerunning{Abbreviated paper title}
% If the paper title is too long for the running head, you can set
% an abbreviated paper title here
%
\author{Yifei Fan}
%
\authorrunning{Y. Fan et al.}
% First names are abbreviated in the running head._R
% If there are more than two authors, 'et al.' is used.
%
\institute{The Hong Kong University of Science and Technology, Hong Kong SAR,China\\
    \email{yfanba@ust.hk}}
%
\maketitle              % typeset the header of the contribution

\section{Introduction}
In real world, more and more The Internet of Things participate in our life 
delivering innovative applications which make our life better. However, those IoT devices 
rely on robust and reliable communication wireless network to work well with other IoT devices.
Thus, the communication resource allocation problem becomes the basic problem the IoT system should face.
There are many diffierent approches to solve this problem. One of these approches is game-theortic approach, 
people can utilize this approch to enable IoT devices to allocate communication resources in a self-organized manner.

In this work, we are going to explore an approch based on Kolkata paise restaurant game\cite{ref:KPR} framework to implement communication 
resources allocations among IoT devices and improve the usage of resource units by utilizing the information from neighbors to avoid confliction.

\section{Theory}

\subsection{IoT System Model}

Consider a wireless IoT System having $b$ resource blocks to serves $N$ IoT devices. 
Each resource block(RB) can serve only one IoT device at a time slot. Whenever multiple IoT devices
try to use the same RB, the RB will fail to serve due to confliction. 
During a time slot, each device may try to send information with probability $p$, 
thus the expected number of IoTs in sending status in a time slot is $N\cdot p$.

\subsection{Random Resource Allocation}

A simple approach is just to allow the IoT devicesto randomly choose RBs. 
In this case, $b$ RBs are chosen with the equal probability. 
Let $S$ be the random variable of the number of IoT devices who are served by RB successfully.
Let $N_t$ the number of IoT devices who sends information is $N_t$ at time slot $t$.
Then the conditional probability distribution function of $S$ at time slot $t$ when $N_t=n_t$,

\begin{equation}
\label{eq:pdfS}
P(S\geq s|N_t=n_t)=\prod_{k=0}^{s-1}\frac{b-k}b\left(\frac{b-s}{b}\right)^{(n_t-s)}, s\in[0,b-1].
\end{equation}

By using \eqref{eq:pdfS}, one can obtain the probability function of $P(S=s|N_t=n_t)$.

The probability function of $N_t$ is:

\begin{equation}
\label{eq:pdfNt}
P(N_t=n_t)=\binom{N}{n_t}p^{n_t}(1-p)^{N-n_t}.
\end{equation}

Then, we can obtain the unconditional probability function $S$,

\begin{equation}
P(S=s)=\sum_{n_t}^N\binom{n_t}{s}P(S=s|N_t=n_t)P(N_t=n_t)
\end{equation}

Then, we can obtain the expectation of $S$
\begin{equation}
\label{eq:ES}
E[S]=\sum_{s=1}^bsP(S=s)
\end{equation}

All of above formula are from the paper\cite{ref:paper}.

Now, in this report, we give a approximate formula to estimate the service rate of one RB.
Suppose the service rate of each resource block is independent, 
then we can calculate the service of one RB by

\begin{equation}
E[\mbox{Service Rate}]=\frac nb\left(\frac{b-1}{b}\right)^{n-1}
\end{equation}

which is just a estimation formula. The therotical formula for service rate is $\frac{E[S]}{b}$, one can figure it out using equation \eqref{eq:ES}.

\subsection{Resource Allocation based on KPR Game}

The IoT devices can utilize information of the RB usage of other IoTs at time slot $t-1$ to
make decision to choose RB intelligently at time slot $t$. However, in practical situation, 
accessing information from other IoTs also involves the communications among IoTs which leads to the additional cost.
Thus, we should limit the communication range among IoTs to reduce the cost into an affordable number. 
What's more, from therotical analysis, if the IoTs can obtain all the information of all IoTs, then they will make the same choice,
which will lead to a bad service rate of RBs.

In this model, the IoT devices can only communicate with neighbor IoTs within a
range $r_c$ to learn about their RB usage at time slot $t-1$. Let $\tau_t$ be the set
IoTs who send information at time slot $t$ and $\mathcal{N}_i$ be the set of neighbors of IoT$_i$.
Let $S_t$ be the set of IoTs that transmitted successfully at slot $t$ and $F_t$ the set of IoT
devices that transmit u successfully at slot $t$ such that $S_t \bigcup F_t = \tau_t$ and $S_t\bigcap F_t = \emptyset$. 
Let $r_{t,i}$ denote the RB used by IoT$_i$ in $\tau_t$. Assume each RB has a rank and all IoTs perfer the RB with higher rank.

The algorithm is shown in Algorithm. \ref{alg:rbc}.

\begin{algorithm}
    \caption{Resource Block Choosing Algorithm}
    \label{alg:rbc}
    \begin{algorithmic}
    \If{$F_{t-1}\bigcap \mathcal{N}_i\neq \emptyset$}
        \State $r_{t,i} \gets$ random(RBs whose rank $\leq$ RBs used by $F_{t-1}\bigcap \mathcal{N}_i$)
    \ElsIf{$S_{t-1}\bigcap \mathcal{N}_i\neq \emptyset$}
        \State $r_{t,i} \gets$ random(RBs whose rank $\geq$ RBs used by $S_{t-1}\bigcap \mathcal{N}_i$)
    \Else
        \State $r_{t,i} \gets$ random(RBs)
    \EndIf
    \end{algorithmic}
\end{algorithm}

\subsubsection{Understanding of this algorithm}
\hfill\break
Let we explain why this algorithm really makes sense. 

Assume you are the IoT$_i$, when your neighbor IoT$_j$ failed with RB$_k$, 
which means the RBs with higher rank than the rank of RB$_k$ 
have relative high probability to be taken in the next time slot, thus you should try other RB with lower rank to avoid confliction with other IoTs.

Similarly, when your neighbor IoT$_j$ succeed with RB$_k$, 
which means the RBs with lower rank than the rank of RB$_k$ 
are easy to access withou confliction in the next time slot, thus you should try other RB with higher rank in order to get better communication performace.

\section{Simulations}

\subsection{Configure}

We are going to carry out a computer simulation to compare the performace of Random Resource Allocation
and the Resource Allocation based on KPR Game. 
We assume that the IoT devices are deployed within a square field of dimension $R$ meters
based on Poisson point process with device density $λ$ persquare meter. 
In our simulations, we set the dimension $R$ of the
deployment region to be 20 meters and the number of resource
blocks $b$ to be 5. The communication ranges $r_c$ will be varied. Furthermore, we analyze the performance of the
learning framework with low device density of $\lambda = 2.5$ and
high device density of $\lambda = 5$. Moreover, we study the effect
of the transmission probabilities $p$ on the performance of two method. 

\subsection{Result and analysis}

The experiment result of situation of transmission probabilities $p=0.01$ is shown in Fig.\ref{fig:result1}.

\begin{figure}
    \centering
    \includegraphics[width=0.8\textwidth]{../figs/Average service rate of RBs (p=0.01).pdf}
    \caption{The simulation result of service rate of RBs using two method. In this experiment, we set $p=0.01$.} 
    \label{fig:result1}
\end{figure}

From the Fig.\ref{fig:result1}, we can see the method of Algorithm. \ref{alg:rbc} can improve the service rate of RBs with specified parameter of 
communication range. 

For the situation of $\lambda=2.5$, the enhancement of proposed algorithm is not big, but it indeed improve a little bit. 
The highest service rate can be achieved when $r_c\approx3$.

For the situation of $\lambda=5$, the enhancement of proposed algorithm is significant.
The highest service rate can be achieved is $28.14\%$ when $r_c\approx3.2$, which almost four times greater than baseline.

However, the parameter of communication range $r_c$ is not the bigger the better. 
You can see the performance drops when $r_c$ is too big. 
Since when $r_c$ is too big, IoT can get information from almost all other IoTs, then they make the same chioce which lead to confliction.
Thus, the parameter of $r_c$ should tune when apply this algorithm.

The experiment result of situation of transmission probabilities $p=0.05$ is shown in Fig.\ref{fig:result2}.

\begin{figure}
    \centering
    \includegraphics[width=0.8\textwidth]{../figs/Average service rate of RBs (p=0.05).pdf}
    \caption{The simulation result of service rate of RBs using two method. In this experiment, we set $p=0.05$.} 
    \label{fig:result2}
\end{figure}

The enhancement of proposed algorithm is more significant.
Since when $p$ is larger enough, the $E[N_t]$ will be big, 
the random resource allocation will lead massive confliction then reduce the service rate.
On the contrary, the proposed algorithm can allocate the resource blocks more efficiently, 
then reduce the probability of confliction and improve the performace.

\newpage
\begin{thebibliography}{8}
    \bibitem{ref:KPR}
    Ghosh, Asim, et al. "Kolkata paise restaurant problem: An introduction." Econophysics of Systemic Risk and Network Dynamics. Springer, Milano, 2013. 173-200.

    \bibitem{ref:paper}
    Park, Taehyeun, and Walid Saad. "Kolkata paise restaurant game for resource allocation in the Internet of Things." 2017 51st Asilomar Conference on Signals, Systems, and Computers. IEEE, 2017.

    \bibitem{ref2}
    Chakrabarti, Anindya Sundar, et al. "The Kolkata Paise Restaurant problem and resource utilization." Physica A: Statistical Mechanics and its Applications 388.12 (2009): 2420-2426.
\end{thebibliography}

\section*{Appendix}

\subsection*{Data of result of simulation}

\subsubsection*{Data of experiment($p=0.01$)}

\hfill\break
\csvautotabular{../out/data1.csv}

\subsubsection*{Data of experiment($p=0.05$)}

\hfill\break
\csvautotabular{../out/data2.csv}


% \begin{figure}
%     \centering
%     \includegraphics[width=0.8\textwidth]{../figs/simple.pdf}
%     \caption{The trajectories of first 20 particles of active Brownian motion with specifid parameter.} 
%     \label{fig:trajectories}
% \end{figure}

\newpage
\subsection*{Code for simulation}

\lstinputlisting[language=Julia]{../code/KPR4IoT.jl}
    
\end{document}
