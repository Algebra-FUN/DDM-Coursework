\documentclass[runningheads]{llncs}

\usepackage{amsmath}
\usepackage{graphicx}
\usepackage[T1]{fontenc}
\usepackage{beramono}
\usepackage{listings}
\usepackage[usenames,dvipsnames]{xcolor}
\usepackage[colorlinks=true, linkcolor=blue]{hyperref}

%%
%% Julia definition (c) 2014 Jubobs
%%
\lstdefinelanguage{Julia}%
  {morekeywords={abstract,break,case,catch,const,continue,do,else,elseif,%
      end,export,false,for,function,immutable,import,importall,if,in,%
      macro,module,otherwise,quote,return,switch,true,try,type,typealias,%
      using,while},%
   sensitive=true,%
   alsoother={$},%
   morecomment=[l]\#,%
   morecomment=[n]{\#=}{=\#},%
   morestring=[s]{"}{"},%
   morestring=[m]{'}{'},%
}[keywords,comments,strings]%

\lstset{%
    language         = Julia,
    basicstyle       = \ttfamily,
    keywordstyle     = \bfseries\color{blue},
    stringstyle      = \color{magenta},
    commentstyle     = \color{ForestGreen},
    showstringspaces = false,
}

\definecolor{codegreen}{rgb}{0,0.6,0}
\definecolor{codegray}{rgb}{0.5,0.5,0.5}
\definecolor{codepurple}{rgb}{0.58,0,0.82}
\definecolor{backcolour}{rgb}{0.95,0.95,0.92}

\lstdefinestyle{mystyle}{
    backgroundcolor=\color{backcolour},   
    commentstyle=\color{codegreen},
    keywordstyle=\color{magenta},
    numberstyle=\tiny\color{codegray},
    stringstyle=\color{codepurple},
    basicstyle=\ttfamily\footnotesize,
    breakatwhitespace=false,         
    breaklines=true,                 
    captionpos=b,                    
    keepspaces=true,                 
    numbers=left,                    
    numbersep=5pt,                  
    showspaces=false,                
    showstringspaces=false,
    showtabs=false,                  
    tabsize=2
}

\lstset{style=mystyle}

% Used for displaying a sample figure. If possible, figure files should
% be included in EPS format.
%
% If you use the hyperref package, please uncomment the following line
% to display URLs in blue roman font according to Springer's eBook style:
% \renewcommand\UrlFont{\color{blue}\rmfamily}

\begin{document}
%
\title{Stochastic simulation for active Brownian particles}
%
%\titlerunning{Abbreviated paper title}
% If the paper title is too long for the running head, you can set
% an abbreviated paper title here
%
\author{Yifei Fan}
%
\authorrunning{Y. Fan et al.}
% First names are abbreviated in the running head._R
% If there are more than two authors, 'et al.' is used.
%
\institute{The Hong Kong University of Science and Technology, Hong Kong SAR,China\\
    \email{yfanba@ust.hk}}
%
\maketitle              % typeset the header of the contribution

\section{Introduction}
Self-propelled  Brownian  particles  have  come  under  the  spotlight  of  the  physical  and  biophysical 
research communities. These active particles are biological or man-made microscopic and nanoscopic 
objects that can propel themselves by taking up energy from their environment and converting it into 
directed motion. On the one hand, self-propulsion is a common feature in microorganisms and allows 
for a more efficient exploration of the environment when looking for nutrients or running away from 
toxic substances. On the other hand, tremendous progress has recently been made toward the fabrication 
of  artificial  microswimmers  and  nanoswimmers  that  can  self-propel  based  on  different  propulsion 
mechanisms\cite{ref:ap}. 

In this article, a series of simulation will be carried out to reseach the behaviour of active Brownian particles.

\section{Model of active Brownian motion}
Consider  a  self-propelled  particle  with  velocity $v$.  The  direction  of  motion  is  subject  to  rotational
diffusion,  which  leads  to  a  coupling  between  rotation  and  translation.  The  corresponding  stochastic
differential equations in 2-dimensional space are
\begin{equation}
    \label{eq:xode}
    \frac{dx}{dt}=v\cos{\phi},
\end{equation}
\begin{equation}
    \label{eq:yode}
    \frac{dy}{dt}=v\sin{\phi},
\end{equation}
\begin{equation}
    \label{eq:phiode}
    \frac{d\phi}{dt}=\sqrt{2D_R}\xi.
\end{equation}
Here the particle position is at $\vec{r}(t)=\left(x(t),y(t)\right)$
and the direction of motion is $\vec{n}(t)=\left(\cos{\phi(t)},\sin{\phi(t)}\right)$
in two dimensions. The active velocity $v$ is a positive constant, $D_R$ is the
rotational diffusion coefficient, and $\xi$ is a Gaussian white noise satisfying $\left\langle\xi\right\rangle$
and $\left\langle\xi(t_2)\xi(t_1)\right\rangle=\delta(t_2-t_1)$
(i.e. with zero mean and correlation $\delta(t)$).
\subsection{Description of its dynamics}
Consider the dynamics governed by equations \eqref{eq:xode}, \eqref{eq:yode}, and\eqref{eq:phiode}.\\
The equations \eqref{eq:xode} and \eqref{eq:yode} describe the change of the particle position,
it only depends on constant velocity rate $v$ and stochastic variational angle of direction $\phi$.
Thus, one can easily obtain $x(t)$ and $y(t)$ by solving these ODE with initial condition $x(0)=x_0, y(0)=y_0$ and a determinate given $\phi(t)$ in this way:
$$
    x(t)=x_0+v\int_0^t\cos\phi(\tau)d\tau,
$$
$$
    y(t)=y_0+v\int_0^t\sin\phi(\tau)d\tau.
$$
The equation \eqref{eq:phiode} describe the change of angle of direction $\phi$ of the particle
and the $\frac{d\phi}{dt}$ only depends on a Gaussian white noise $\xi$.
Thus, one can obtain $\phi(t)$ by solving this Stochastic ODE with initial condition $\phi(0)=\phi_0$ in this way:
$$
    \phi(t)=\phi_0+\sqrt{2D_R}\int_0^t\xi(\tau)d\tau.
$$

\subsection{Autocorrelation of the direction of motion}
The autocorrelation of the direction of motion is given by
\begin{equation}
    \label{eq:autocorr}
    \left\langle\vec{n}(s+t)\cdot\vec{n}(s)\right\rangle = \lim_{T\to\infty}\left[\frac1T\int_0^T\vec{n}(s+t)\cdot\vec{n}(s)ds\right]=\exp\left(-\frac{\left\lvert t\right\rvert }{\tau_R}\right)
\end{equation}
where $\tau_R$ is  the  orientational  persistence  time.\\
The equation \eqref{eq:autocorr} can be obtained by simple derivation,
$$
    \begin{aligned}
        \left\langle \vec{n}(s+t)\cdot\vec{n}(s)\right\rangle
         & =\left\langle\cos(\phi(s+t)-\phi(s))\right\rangle                            \\
         & =\left\langle\cos(\phi(t)-\phi(0))\right\rangle                              \\
         & =\left\langle\cos\left(\sqrt{2D_R}\int_0^t\xi(\tau)d\tau\right)\right\rangle \\
    \end{aligned}
$$
Denote $\zeta(t)=\int_0^t\xi(\tau)d\tau$, since $\xi$ is a Gaussian white noise, then $\zeta(t)\sim \mathcal{N}(0,t)$.
$$
    \begin{aligned}
        \left\langle \vec{n}(s+t)\cdot\vec{n}(s)\right\rangle
         & =\left\langle\cos\left(\sqrt{2D_R}\zeta(t)\right)\right\rangle                                               \\
         & =\int_{-\infty}^{\infty}\cos\left(\sqrt{2D_R}u\right)\frac1{\sqrt{2\pi t}}\exp\left(-\frac{u^2}{2t}\right)du \\
         & =\exp\left(-D_R t\right)
    \end{aligned}
$$
Since only consider $t>0$ here, in order to adapt for condition $t\leq 0$, then
$$
    \left\langle \vec{n}(s+t)\cdot\vec{n}(s)\right\rangle=\exp\left(-D_R \left\lvert t\right\rvert\right)
$$
which also shows the relation 
\begin{equation}
    \label{eq:depD}
    \tau_R=\frac1{D_R},
\end{equation}
again.

\subsubsection{Understanding of behaviour of autocorrelation of direction of motion}
\hfill\break

The equation \eqref{eq:autocorr} descirbe the autocorrelation of direction of motion,
we can see the behaviour of autocorrelation between $\vec{n}(s)$ and $\vec{n}(s+t)$ with time lag $t$ is exponential decay when time lag $t$ increase.
This exponential decay will be faster, when $\tau_R$ decrease, which shows why $\tau_R$ is called orientational persistence time.
Since when $\tau_R$ increase, autocorrelation of direcation becomes bigger at a given time lag $t$,
then $\max\left\{ t|\left\langle\vec{n}(s+t)\cdot\vec{n}(s)\right\rangle>1-\epsilon\right\}$ becomes bigger,
which means orientation of particle can be persisted in one rough direction in longer time.

\subsection{Mean square displacement}
The mean square displacement is given by
\begin{equation}
    \text{MSD}(t)=\left\langle \left[\vec{r}(t)-\vec{r}(0)\right]^2 \right\rangle=\left\langle\left[x(t)-x(0)\right]^2 \right\rangle+\left\langle\left[y(t)-y(0)\right]^2 \right\rangle,
\end{equation}
which is analytically given by 
\begin{equation}
    \label{eq:MSD}
    \text{MSD}(t)=2v^2\tau_R t,
\end{equation}
for $t\gg \tau_R$.

\subsubsection{Understanding of behaviour of MSD(t)}
\hfill\break

The MSD descirbe how far the particle can reach, it naturally depends on $v,t$ without need to explain.
Meanwhile, it also depends on $\tau_R$, as it's explained in previous section,
it represents the orientational persistence time, thus with long time to persist in one rough direction,
the particle can run longer in this direction,
otherwise it just runs in a circle when $D_R$ is big which means with high change rate of direction.
\section{Simulation}

\subsection{Numerical solution of this Stochastic ODE}

In order to simulation the dynamics progress of active Brownian motion on computer,
use Euler forward method to obtain numerical solution for Stochastic ODE expressed in equations \eqref{eq:xode}, \eqref{eq:yode}, and\eqref{eq:phiode}.\\
First of all, convert equation \eqref{eq:phiode} into difference equation with time interval $\delta t$,
\begin{equation}
    \phi(t+\Delta t)=\phi(t)+\sqrt{2D_R}\int_0^{\Delta t}\xi(\tau)d\tau,
\end{equation}
where $\int_0^{\Delta t}\xi(\tau)d\tau\sim\mathcal{N}(0,\Delta t)$.
In each time step, generate a random number $\zeta\sim \mathcal{N}(0,\Delta t)$ to represent $\int_0^{\Delta t}\xi(\tau)d\tau\sim\mathcal{N}(0,\Delta t)$.\\
Then,
\begin{equation}
    \label{eq:phidif}
    \phi(t+\Delta t)=\phi(t)+\sqrt{2D_R}\zeta.
\end{equation}
For equations \eqref{eq:xode} and \eqref{eq:yode}, they can be converted into difference equations in this form:
\begin{equation}
    \label{eq:xdif}
    x(t+\Delta t)=x(t)+v\cos(\phi(t)),
\end{equation}
\begin{equation}
    \label{eq:ydif}
    y(t+\Delta t)=y(t)+v\sin(\phi(t)).
\end{equation}
In the following simulation, these three equations \eqref{eq:phidif}, \eqref{eq:xdif} and \eqref{eq:ydif} will be used to carry simulation with Julia language. 
For simplification, set initial condition $x(0)=0,y(0)=0,\phi(0)=0$ for all particles for simplification.\\
The simulation result for trajectories of particles can be seen in Fig.~\ref{fig:trajectories} in Appendix, which can give people some basic intuition of this active Brownian motion.

% \subsection{Simple simulation}

% In this section, just to carry out a simulation with simple parameter to get intuition of this active Brownian motion of particles
% by observing their trajectories.

% \noindent For simplification, set initial condition $x(0)=0,y(0)=0$ for all particles for simplification, set $\phi(0)$ by a random angle which satisfies a uniform distribution $\mathcal{U}(0,2\pi)$ 
% and set the totoal number of particles $N=1000$, end time $T_{end}=50$, time step $\Delta t=0.1$ and $D_R=1, v=1$.

% \begin{figure}
%     \centering
%     \includegraphics[width=0.6\textwidth]{../figs/simple.pdf}
%     \caption{The trajectories of first 20 particles of active Brownian motion with specifid parameter.} 
%     \label{fig:trajectories}
% \end{figure}

% From Fig.~\ref{fig:trajectories}, we can see the particles swimming in the two dimensional space
% and their trajectories seems irregular, but people still can find pattern from some statistics such as 
% autocorrelation and MSD which we are going to research in the following simulations.

\subsection{Simulation to verify autocorrelation}

This simulation is going to be carried out to verify whether autocorrelation satisfies equation \eqref{eq:autocorr}. \\
The $\left\langle\cos(\phi(s+t)-\phi(s))\right\rangle$ will be numerically computed and 
fitted to the exponential function \eqref{eq:autocorr} to estimate $\tau_R$, denote the estimated value as $\hat{\tau}_R$.\\
First of all, compute the estimated autocorrelation $\hat{C}(t)$ with
\begin{equation}
\begin{aligned}
\hat{C}(t)
&=\left\langle\cos(\phi(s+t)-\phi(s))\right\rangle\\
&=\frac1N\sum_{k=1}^N\frac1M\sum_{i=1}^M\cos(\phi_k(t_i+t)-\phi_k(t_i)),
\end{aligned}
\end{equation}
where $\{\phi_k(t)\}_{k=1}^N$ is the simulated $\phi(t)$ for the $k^{th}$ particle, $N$ is the total number of particle
and $\{t_i\}_{i=1}^M$ is a series of chosen time knots which satisfying $t_i-t_{i-1}=\Delta t$, $M$ is the number of time knots.\\
Then the estimated $\hat{\tau}_R$ can be computed by
\begin{equation}
\hat{\tau}_R=-\frac1{\left({T^\top T}\right)^{-1}T^\top \hat{C}},
\end{equation}
where $T$ is vector formed by $\{t_i\}_{i=1}^M$ 
and $\hat{C}$ is also a vector formed by $\{\hat{C}(t_i)\}_{i=1}^M$.

\noindent For this simulation, let set the totoal number of particles $N=1000$, 
for simplification set initial condition $x(0)=0,y(0)=0,\phi(0)=0$ for all particles,
a set of selected values of $D_R=\{0.01,0.05,0.1,0.5,1,5,10,50,100\}$ or $\text{logspace}(-2,2,50)$,
set time interval $\Delta t=10^{-3}$ and stop time $T_{end}=10\cdot\tau_R=10/D_R$.

\subsubsection*{Simulation result of $\hat{C}(t)$}
\hfill\break

\begin{figure}
    \centering
    \includegraphics[width=0.8\textwidth]{../figs/Verify_autocorr.pdf}
    \caption{The simulation result of estimated autocorrelation of direction of motion $\hat{C}(t)$.} \label{fig:corr}
\end{figure}

\noindent The simulation results of estimated autocorrelation of direction of motion $\hat{C}(t)$ for each $D_R=\{0.01,0.05,0.1,0.5,1,5,10,50,100\}$ 
and corresponding estimated $\hat{\tau}_R$
are shown in Fig.~\ref{fig:corr}, which shows the simulation result is quite close to the therotical values.
Hence, this simulation can verify that autocorrelation of direction of motion satisfies \eqref{eq:autocorr}.

\newpage
\subsubsection*{Simulation result of dependence of $\tau_R$ on $D_R$}
\hfill\break

\begin{figure}
    \centering
    \includegraphics[width=0.55\textwidth]{../figs/dependence of τ and D.pdf}
    \caption{The dependence of $\tau_R$ on $D_R$.} \label{fig:depD}
\end{figure}

\noindent The simulation result of the dependence of $\tau_R$ on $D_R$ based on a set of selected value $D_R=\text{logspace}(-2,2,50)$ is shown in Fig.~\ref{fig:depD},
which uses log-log plot and shows that simulated $\hat{\tau}_R$ satisfies the equation \eqref{eq:depD}.

\subsection{Simulation to verify MSD}
This simulation is going to carried out to verify whether $\text{MSD}(t)$ satisfies equation \eqref{eq:MSD} when $t\gg\tau_R$.\\
The estimate $\widehat{\text{MSD}}(t)$ is computed by
\begin{equation}
    \widehat{\text{MSD}}(t)=\frac1N\sum_{k=1}^N\left(\left[x_k(t)-x_k(0)\right]^2+\left[y_k(t)-y_k(0)\right]^2\right),
\end{equation} 
where $\{x_k(t)\}_{k=1}^N$ and $\{y_k(t)\}_{k=1}^N$ is the series of the simulated position of particle.

\noindent For this simulation, let set the totoal number of particles $N=1000$, 
for simplification set initial condition $x(0)=0,y(0)=0,\phi(0)=0$ for all particles, 
choose a set of values of $v=\{0.1,1,10\}$ and $\tau_R=\{0.01,0.1,1,10,100\}$,
set time interval $\Delta t=10^{-3}$ 
and stop time $T_{end}=50\cdot\tau_R$ to satisfy condition $t\gg\tau_R$.
\subsubsection*{Simulation result}
\hfill\break

\begin{figure}
    \includegraphics[width=\textwidth]{../figs/Verify_MSD.pdf}
    \caption{The simulation result of function $\widehat{\text{MSD}}(t)$.} \label{fig:MSD}
\end{figure}

\noindent The simulation result of $\widehat{\text{MSD}}(t)$ is shown in Fig.~\ref{fig:MSD}, 
which shows the estmiated line $\widehat{\text{MSD}}(t)$ is quite close to the therotical line $\text{MSD}(t)$.
Hence, this simulation can verify the function $\text{MSD}(t)$ satisfies the equation \eqref{eq:MSD}.

% \begin{table}
%     \caption{Table captions should be placed above the
%         tables.}\label{tab1}
%     \begin{tabular}{|l|l|l|}
%         \hline
%         Heading level     & Example                                          & Font size and style \\
%         \hline
%         Title (centered)  & {\Large\bfseries Lecture Notes}                  & 14 point, bold      \\
%         1st-level heading & {\large\bfseries 1 Introduction}                 & 12 point, bold      \\
%         2nd-level heading & {\bfseries 2.1 Printing Area}                    & 10 point, bold      \\
%         3rd-level heading & {\bfseries Run-in Heading in Bold.} Text follows & 10 point, bold      \\
%         4th-level heading & {\itshape Lowest Level Heading.} Text follows    & 10 point, italic    \\
%         \hline
%     \end{tabular}
% \end{table}

% %
% % ---- Bibliography ----
% %
% % BibTeX users should specify bibliography style 'splncs04'.
% % References will then be sorted and formatted in the correct style.
% %
% % \bibliographystyle{splncs04}
% % \bibliography{mybibliography}
% %
\begin{thebibliography}{8}
    \bibitem{ref:ap}
    Bechinger, C., Di Leonardo, R., Löwen, H., Reichhardt, C., Volpe, G., \& Volpe, G. (2016). Active 
    particles in complex and crowded environments. Reviews of Modern Physics, 88(4), 045006. 

    \bibitem{ref:abp}
    Romanczuk, P., Bär, M., Ebeling, W., Lindner, B., \& Schimansky-Geier, L. (2012). Active Brownian 
    particles. The European Physical Journal Special Topics, 202(1), 1-162. 
\end{thebibliography}

\section*{Appendix}

\subsection*{Other figures}

\begin{figure}
    \centering
    \includegraphics[width=0.8\textwidth]{../figs/simple.pdf}
    \caption{The trajectories of first 20 particles of active Brownian motion with specifid parameter.} 
    \label{fig:trajectories}
\end{figure}

\subsection*{Code for simulation}

\lstinputlisting[language=Julia]{../simulation.jl}
    
\end{document}
