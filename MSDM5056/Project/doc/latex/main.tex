\documentclass[runningheads]{llncs}

\usepackage{amsmath}
\usepackage{graphicx}
\usepackage[T1]{fontenc}
\usepackage{beramono}
\usepackage{listings}
\usepackage[usenames,dvipsnames]{xcolor}
\usepackage[colorlinks=true, linkcolor=blue]{hyperref}

% Used for displaying a sample figure. If possible, figure files should
% be included in EPS format.
%
% If you use the hyperref package, please uncomment the following line
% to display URLs in blue roman font according to Springer's eBook style:
% \renewcommand\UrlFont{\color{blue}\rmfamily}

\begin{document}
%
\title{Project: Community Detection and Network Resilience}
%
%\titlerunning{Abbreviated paper title}
% If the paper title is too long for the running head, you can set
% an abbreviated paper title here
%
\author{Yifei Fan}
%
\authorrunning{Y. Fan et al.}
% First names are abbreviated in the running head._R
% If there are more than two authors, 'et al.' is used.
%
\institute{The Hong Kong University of Science and Technology, Hong Kong SAR,China\\
    \email{yfanba@ust.hk}}
%
\maketitle              % typeset the header of the contribution

\section{Dataset}
In this report, we chooose "Coauthorships in network science" dataset\cite{ref:dataset} as our research target.

\subsection{Intro}

This "Coauthorships in network science" dataset contains a coauthorship network of scientists
working on network theory and experiment, as compiled by M. Newman in May
2006.  The network was compiled from the bibliographies of two review
articles on networks, M. E. J. Newman, SIAM Review 45, 167-256 (2003) and
S. Boccaletti et al., Physics Reports 424, 175-308 (2006), with a few
additional references added by hand.  The version given here contains all
components of the network, for a total of 1589 scientists, and not just the
largest component of 379 scientists previously published.

\section{Exploration on it's basic feature}

The network visualization can be seen in Fig.\ref{fig:vis}.

\begin{figure}
    \centering
    \includegraphics[width=0.8\textwidth]{../../fig/network.pdf}
    \caption{Network Visualization of "Coauthorships in network science" network} 
    \label{fig:vis}
\end{figure}

\subsection{Degree distribution}

The degree distribution plot can be seen at Fig.\ref{fig:deg-dist}.
From this figure, we can see the degree distribution "Coauthorships in network science" network obeys
the power law, which means it can be seen as a scale-free network.

\begin{figure}
    \centering
    \includegraphics[width=\textwidth]{../../fig/deg-dist.pdf}
    \caption{Degree distribution of "Coauthorships in network science" network} 
    \label{fig:deg-dist}
\end{figure}

\subsection{Average shortest distance}

Since this network isn't a connected graph, even some node are isolated,
so in order to finish this task, we just compute it's average shortest distance on it's giant component.
The result of average shortest distance of giant component of the network is about $6.041867347935949$,
which accords with the six degree of separation theory.

\subsection{Average clustering coefficient}

The result is $0.6377905695067805$.

\section{Communiy Detection}

Use the Girvan-Newman algorithm\cite{ref:gn-algo} to find its communities.

The result is given by following Fig.\ref{fig:community}. 
Since this network isn't a connected network, so nodes get into a lot of different communities about 400
communities. But there are two major communities, the nodes in these two major communities are colored in orange and blue.
As the algorithm progresses, these communities also split and some nodes form new communities.

\begin{figure}
    \centering
    \includegraphics[width=\textwidth]{../../fig/communities.pdf}
    \caption{Visualization of community detection of "Coauthorships in network science" network using Girvan-Newman algorithm. 
    In this figure, we only show the communities whose number of members is greater than 30 and obmit other small community. 
    Here $C$ means the number of communities.} 
    \label{fig:community}
\end{figure}

\section{Network Resilience}

We perform a series of random attack and a targeted attack on the network
by removing nodes from the network with different node removal percentage $p$
and observe the variation of the size of the giant component in this network.

In random attack, we remove $\left\lfloor p N\right\rfloor$ nodes randomly from the network.
In targeted attack, we remove nodes with $\left\lfloor p N\right\rfloor$ largest degree.

We carry out 50 times random attack and report their average in each experiment.
And, change parameter $p$ from 0 to 1.

The experiment results are shown in Fig.\ref{fig:attack}.

\begin{figure}
    \centering
    \includegraphics[width=0.8\textwidth]{../../fig/attack.pdf}
    \caption{Size of giant component of "Coauthorships in network science" network after being attack by remove nodes.} 
    \label{fig:attack}
\end{figure}

From experiment result, we can conclude that the "Coauthorships in network science" network 
is resilience to random attacks, but it's very sensitive to targeted attacks
, which shows that the scientific research relys on some small 
amount of important scientist to intergate whole science community to work together.

% \begin{table}
%     \caption{Table captions should be placed above the
%         tables.}\label{tab1}
%     \begin{tabular}{|l|l|l|}
%         \hline
%         Heading level     & Example                                          & Font size and style \\
%         \hline
%         Title (centered)  & {\Large\bfseries Lecture Notes}                  & 14 point, bold      \\
%         1st-level heading & {\large\bfseries 1 Introduction}                 & 12 point, bold      \\
%         2nd-level heading & {\bfseries 2.1 Printing Area}                    & 10 point, bold      \\
%         3rd-level heading & {\bfseries Run-in Heading in Bold.} Text follows & 10 point, bold      \\
%         4th-level heading & {\itshape Lowest Level Heading.} Text follows    & 10 point, italic    \\
%         \hline
%     \end{tabular}
% \end{table}

% %
% % ---- Bibliography ----
% %
% % BibTeX users should specify bibliography style 'splncs04'.
% % References will then be sorted and formatted in the correct style.
% %
% % \bibliographystyle{splncs04}
% % \bibliography{mybibliography}
% %
\begin{thebibliography}{8}
    \bibitem{ref:dataset}
    Newman, M. E. (2006). Finding community structure in networks using the eigenvectors of matrices. Physical review E, 74(3), 036104.

    \bibitem{ref:gn-algo}
    Girvan, M., Newman, M. E. J.: Community structure in social and biological 
    networks. Proc. Natl. Acad. Sci. USA 99(12), 7821–7826 (2002). 
\end{thebibliography}

\section*{Appendix}

\subsection*{Code for simulation}

\end{document}
